
\section{Bayesian inference}

\subsection{Introduction to Bayes' theorem}
Bayes' theorem is a formula that describes how to update the probabilities of hypotheses when given evidence. It follows simply from the axioms of conditional probability, but can be used to powerfully reason about a wide range of problems involving belief updates.

Given a hypothesis $H$ and evidence $E$, Bayes' theorem states that the relationship between the probability of the hypothesis $Pr(H)$, before getting the evidence, and the probability of the hypothesis after getting the evidence $Pr(H|E)$ is
\begin{equation}
Pr(H|E) = \frac{Pr(E|H)Pr(H)}{Pr(E)}
\end{equation}
Often there are competing hypothesis and the task is to determine which is the most probable.

This formula relates the probability of the hypothesis before getting the evidence $Pr(H)$, to the probability of the hypothesis after getting the evidence, $Pr(H|E)$: this term is generally what we want to know. For this reason, $Pr(H)$ is called the \textbf{prior probability}, while 
$Pr(H|E)$ is called the \textbf{posterior probability}. The factor that relates the two, $\frac{Pr(E|H)}{Pr(E)}$, is called the \textbf{likelihood ratio} and $Pr(E|H)$ is called \textbf{likelihood} which indicates the compatibility of the evidence with the given hypothesis.
$P(H|E)$ and $Pr(E|H)$ are called conditional probabilities. A conditional probability is an expression of how probable one event is given that some other event occurred (a fixed value) and Bayes' theorem centers on relating different conditional probabilities.

$Pr(E)$ is sometimes called \textbf{marginal likelihood} or model evidence this factor is the same for all the hypotheses being considered.

\subsection{Bayes inference}
Bayesian inference is a method of statistical inference in which Bayes' theorem is used to update the probability for a hypothesis as more evidence or information becomes available. Bayesian inference assumes the data were generated by a model with unknown parameters. From this, it tries to come up with beliefs about the likely  "true values" of the parameters of the model. The Bayesian approach differs from the standard ("frequentist") method for inference in its use of a prior distribution to express the uncertainty present before seeing the data, and to allow the uncertainty remaining after seeing the data to be expressed in the form of a posterior distribution.

For this reason when writing down the Bayes theorem in these cases, formally also the conditional on the choice of the model should be written down:

\begin{equation}
Pr(\theta|\X,M) = \frac{Pr(\X|\theta,M)Pr(\theta|M)}{Pr(\X|M)}
\end{equation}
Here the hypothesis described in the previous paragraph is represented by a set of values for the parameters.

Once the model is stipulated, $Pr(X|\theta)$ can be evaluated for any given set of parameters: in this sense the likelihood is fixed once the model is fixed.
$Pr(\X|M)$ can be reexpressed as $Pr(\X|M) =\int Pr(\X|\theta,M)Pr(\theta|M)d\theta$. In this way it can be re-thought as a normalization factor of the sets of parameters since it is a summation over the all parameter space. However note that it \textbf{does depend} on the choice of the model. It can be seen also as asking the question \textit{how much is it probable to see the data we have seen given that the model $M$, without any claim about its parameters, generated those data}?

\subsection{Types of estimation}
\label{ssec:estimations}
It is the moment to clarify different types of estimation. We have already seen the Ordinary Least Square (OLS) estimations, and other types of estimations based on the definition of an error function.

We have also seen the \textbf{Maximum Likelihood Estimation (MLE)} approach, which maximizes the likelihood of the Bayes expression and how it is equivalent to OLS estimation in case of linear regression. (To be precise almost always the log-likelihood is maximized, first of all because of analytical convenience since many times we deal with exponentials coming from Gaussian distribution. Secondly for numerical precision: we are dealing with probabilities, numbers between $0$ and $1$ and since the range is quite small, underflow might be a problem.) The logarithmic instead extends the range by mapping numbers close to $0$ to $\infty$, resulting in a better precision.

The \textbf{Maximum A-Priori (MAP)} estimation maximizes the numerator of the Bayes expression, i.e., the likelihood times the prior, which means the likelihood is weighted by weights coming from the prior. When using a uniform distribution for the prior, MAP turns into MLE since we are assigning equal weights for each possible value. For example suppose we can assign six possible values to $\be$ and assume $P(\be_i)=1/6$:
\begin{equation}
\begin{aligned}
&\be_{MAP} = \argmax{\be}{\sum_i Pr(\x_i|\be) + \log P(\be)} =
\\&= \argmax{\be}{\sum_i Pr(\x_i|\be) +const}=\argmax{\be}{\sum_i Pr(\x_i|\be)} = \be_{MLE}
\end{aligned}
\end{equation}

When using a different prior, i.e., the simplification does not hold anymore.

\subsection{Conjugate distributions}
\label{conjugacy}
\begin{definition}{\textbf{Conjugate distributions}}
In Bayesian probability theory, if the posterior distribution $Pr(\theta|\X,M)$ is in the same probability space as the prior probability distribution $Pr(\theta|M)$, then the prior and posterior are called \textbf{conjugate distributions}, and the prior is called \textbf{conjugate prior for the likelihood function}. 
\end{definition}

As example consider the Gaussian distribution: \textbf{the Gaussian family is conjugate to itself (or self-conjugate) with respect to a Gaussian likelihood function}. If the likelihood is Gaussian, choosing Gaussian prior will ensure that also the posterior distribution is a Gaussian (see \autoref{gaussian}). This means that the Gaussian distribution is a conjugate prior for the likelihood that is also Gaussian.

Consider the general problem of inferring a (continuous) distribution for a parameter $\theta$ given some datum or data $x$. From Bayes' theorem, the posterior distribution is equal to the product of the likelihood function $p(x|\theta, M)$ and the prior $p(\theta|M)$. Let the likelihood function be considered fixed; the likelihood function is usually well-determined from a statement of the data-generating process. It is clear that different choices of the prior distribution $p(\theta|M)$ may make the integral more or less difficult to calculate, and the product $p(x|\theta,M) \times p(\theta|M)$ may take one algebraic form or another. For certain choices of the prior, the posterior has the same algebraic form as the prior (generally with different parameter values). Such a choice is a conjugate prior. A conjugate prior is an algebraic convenience, giving a closed-form expression for the posterior; otherwise numerical integration may be necessary. Further, conjugate priors may give intuition, by more transparently showing how a likelihood function updates a prior distribution.
All members of the exponential family have conjugate priors.

In case of \textbf{classification} we are given a training set with input and output data. Once we have stipulated a model that could have generated output data, we want to find the best parameters of the model such that when those inputs are fed to the model we get those outputs. So the question becomes \textit{what is the best set of parameters $\theta$ for the model $M$ that could have generated the output $\y$ when the model has been fed with input data $\X$?}

In formula:
\begin{equation}
\label{ClassBayes}
Pr(\theta|\y,\X,M) = \frac{Pr(\y|\theta,\X,M)Pr(\theta|\X, M)}{Pr(\y|\X,M)}
\end{equation}

\subsubsection{Dataset likelihood}
\label{dataset likelihood}
\autoref{dataset likelihood}
To be precise the likelihood is the likelihood of an entire dataset, since we are interested in all $\y$ and not in a single value $y$. $Pr(\y|\theta,\X,M)$ is then a joint density over all the responses in our dataset: $Pr(y_1, y_2, \cdots, y_n|\theta,\X,M)$. Evaluating this density at the observed points gives a single likelihood value for the whole dataset.	 Assuming that the noise at each data point is independent we can factorize as:
\begin{equation}
Pr(\y|\theta,\X,M) = \prod_{n=1}^N Pr(y_n|\x_n, \theta)
\end{equation}
We have not say that $y_n$'s are completely independent as otherwise it would not be worth trying to model the data at all. Rather, they are \textbf{conditionally independent} given a value $\theta$, i.e., the deterministic model.
Basically the model incorporates the dependency. Consider the following example, we have a set of data $(\y,\X)$ from which we want to predict the output $y_n$ given a new $\x_n$. Recalling from conditional probability that $P(A|B) = \frac{Pr(A\cap B)}{Pr(B)}$ we have:
\begin{equation}
P(y_n|\y,\x_n, \X) = \frac{Pr(y_n\cap \y|\x_n,\X)}{Pr(\y|\x_n,\X)}
\end{equation}
Note that actually $\y$ does not depend on $\x_n$: $Pr(\y|\x_n,\X) = Pr(\y|\X)$. Using independence:
\begin{equation}
P(y_n|\y,\x_n, \X) = \frac{Pr(y_n\cap \y|\x_n,\X)}{Pr(\y|\X)} = \frac{Pr(\y|\X) Pr(y_n|\x_n,\X)}{Pr(\y|\X)} = 
\end{equation}
[TO BE CONTINUED WAITING ON ANSWER ON CROSS-VALIDATED STACK EXCHANGE]
